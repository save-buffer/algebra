\documentclass{article}
\usepackage{sasha}
\title{Groups}
\author{Sasha Krassovsky}
\begin{document}
\maketitle

\begin{defn}
  A \vocab{group} is a pair $G = (G, \star)$ consisting of a set $G$
  and a binary operation $\star$ on $G$ such that:
  \begin{itemize}
  \item G has an identity $1_G$ or just $1$ such that
    $$\forall g \in G, 1_G \star g = g \star 1_G = g$$
  \item $\star$ is associative, so
    $$\forall a, b, c \in G, (a \star b) \star c = a \star (b \star c)$$
  \item Every element in $G$ has an inverse. That is
    $$\forall g \in G, \exists h \in G, g \star h = h \star g = 1_G$$    
  \end{itemize}
\end{defn}

\begin{remark}
  Notice that $G$ is closed under $\star$ implicitly. That is,
  $\forall g, h \in G, g \star h \in G$. 
\end{remark}

\begin{defn}
  A group is \vocab{abelian} if its operation is commutative ($a \star b = b \star a$).
  Otherwise, it is \vocab{non-abelian}. 
\end{defn}

\begin{defn}
  A mapping is a \vocab{bijection} if it is injective (one-to-one) and
  surjective (onto). 
\end{defn}

\noindent Properties of Groups:
\begin{itemize}
\item Let $G$ be a group.
  \begin{enumerate}
  \item The identity $1_G$ is unique.
  \item The inverse of any element $g \in G$, $g^{-1}$ is unique.
  \item For any $g \in G$, $(g^{-1})^{-1} = g$.
  \end{enumerate}
\item Let $G$ be a group and $a, b \in G$. Then $(ab)^{-1} = b^{-1}a^{-1}$.
\item Let $G$ b ea group and pick a $g \in G$. Then the map $G \rightarrow G$ given
  by $x \mapsto gx$ is a bijection. 
\end{itemize}

\begin{defn}
  Let $G = (G, \star)$ and $H = (H, *)$ be groups. A bijection
  $\phi : G \rightarrow H$ is called an \vocab{isomorphism} if
  $$\forall g_1, g_2 \in G, \phi(g_1 \star g_2) = \phi(g_1) * \phi(g_2)$$
  If there exists an isomorphism from $G$ to $H$, then $G$ and $H$ are \vocab{isomorphic},
  i.e. $G \cong H$. 
\end{defn}

\begin{remark}
  $\cong$ is an equivalence relation (i.e. it is reflexive, symmetric, and transitive). 
\end{remark}

\begin{defn}
  The \vocab{order of a group} $G$ is the number of elements in $G$, denoted $|G|$.
  A group is a \vocab{finite group} if $|G|$ is finite.
\end{defn}

\begin{defn}
  The \vocab{order of an element} $g \in G$ is the smallest positive integer $n$
  such that $g^n = 1_G$ or $\infty$ if no such $n$ exists. Denoted $\ord{g}$. 
\end{defn}

\begin{fact}
  If $G^n = 1_G$ then $\ord{g} | n$. 
\end{fact}

\begin{fact}
  Let $G$ be a finite group. Then for all $g \in G, \ord{g}$ is finite.
\end{fact}

\begin{nthm}[Lagrange's Theorem For Orders]
  Let $G$ be any finite group. Then $\forall x \in G, x^{|G|} = 1_G$ for
  any $x \in G$. This is the general case of Fermat's Little Theorem. 
\end{nthm}

\begin{defn}
  The \vocab{General Linear Group} of degree $n$, denoted $\GL_n$ is the set of
  invertible $n \times n$ matrices along with the operation of matrix
  multiplication. To specify what is in each matrix, we give it an argument.
  For example, the GL over $\RR^{n \times n}$ is denoted $\GL_n(\RR)$. 
\end{defn}

\begin{defn}
  Let $G = (G, \star)$ be a group. A group $H = (H, \star)$ is a
  \vocab{subgroup} of $G$ if $H \subseteq G$. $H$ is called a \vocab{proper subgroup}
  of $G$ if $H \neq G$. 
\end{defn}

\begin{remark}
  If $H$ is a subgroup of $G$, the binary operation is the same. Therefore,
  to specify $H$, you need only provide its elements, not its operation. 
\end{remark}

\begin{defn}
  The \vocab{Special Linear Group} of degree $n$ over a field $F$, denoted $\SL_n(F)$
  is the subgroup of $\GL_n(F)$ such that $\forall x \in \SL_n(F), \det(x) = 1$.
\end{defn}

\begin{defn}
  Let $G$ be a group. Let $S$ be a subset of $G$. The subgroup \vocab{generated} by
  $S$, $\langle S \rangle$ is the set of elements which can be written as a finite
  product of elements in $S$ and their inverses. If $\langle S \rangle = G$,
  then $S$ is a set of \vocab{generators} for G. 
\end{defn}

\begin{defn}
  The \vocab{group presentation} of a group is an expression specifying a
  set of generators and \vocab{relations} between the generators. For example,
  $$\ZZ_{100} = \langle x | x^{100} = 1 \rangle$$
\end{defn}

\begin{remark}
  Determining if a group is finite from its presentation is undecideable.
\end{remark}

\begin{defn}
  Let $G = (G, \star)$ and $H = (H, *)$ be groups. A \vocab{group homomorphism}
  is a map $\phi : G \rightarrow H$ such that
  $$\forall g_1, g_2 \in G, \phi(g_1 \star g_2) = \phi(g_1) * \phi(g_2)$$
  Notice that this is is the same as an isomorphism, only lacking the requirement that
  $\phi$ be a bijection. 
\end{defn}

\begin{defn}
  The \vocab{trivial homomorphism} $G \rightarrow H$ sends every element of $G$ to $1_H$.
\end{defn}

\begin{fact}
  Let $\phi : G \rightarrow H$ be a homomorphism. Then $\phi(1_G) = 1_H$ and
  $\phi(g^{-1}) = \phi(g)^{-1}$.
\end{fact}

\begin{defn}
  The \vocab{kernel} of a homomorphism $\phi : G \rightarrow H$ is a subgroup of $G$ such that
  $$\ker \phi = \{g \in G : \phi(g) = 1_H\}$$
\end{defn}

\begin{defn}
  The \vocab{image} of a homomorphism $\phi : G \rightarrow H$ is a subgroup of $G$ such that
  $$\img \phi = \phi(G) = \{\phi(x) : x \in G \}$$
\end{defn}

\begin{prop}
  The map $\phi$ is injective if and only if $\ker \phi = {1_G}$. 
\end{prop}

\begin{defn}
  Let $H$ be any subgroup of $G$. A set of the form $gH$ for any $g \in G$ is called a
  \vocab{left coset} of $H$. 
\end{defn}

\begin{remark}
  It is possible for $g_1N = g_2N$, even if $g_1 \neq g_2$. 
\end{remark}

\begin{remark}
  Given cosets $g_1H$ and $g_2H$, the map $x \mapsto g_2g_1^{-1}$ is a bijection, so
  all cosets have equal cardinality. 
\end{remark}

\begin{defn}
  A subgroup $N$ of $G$ is called \vocab{normal} if it is the kernel of some surjective
  homomorphism. We write this as $N \trianglelefteq G$. Other words for normal are:
  \vocab{self-conjugate} or \vocab{invariant}. 
\end{defn}

\begin{defn}
  Let $N \trianglelefteq G$. Then the \vocab{quotient group}, denoted $G/N$ is
  the group defined such that:
  \begin{itemize}
  \item The elements of $G/N$ will be the left cosets of $N$.
  \item Let $g_1, g_2 \in G$. Then $(g_1N) \cdot (g_2N) = (g_1g_2)N$. 
  \end{itemize}
\end{defn}

\begin{remark}
  We can define an equivalence relation $\sim_N$ on $G$ by saying $x \sim_N y$ for
  $\phi(x) = \phi(y)$. $\sim_N$ divides $G$ into equivalence classes which are
  in bijection to $G/N$. 
\end{remark}

\begin{nthm}[Lagrange's Theorem]
  Let G be a finite group, and let H be any subgroup. Then $|H|$ divides $|G|$. 
\end{nthm}
\begin{proof}
  Cosets of $H$ have the same size and form a partition on $G$. If $n$ is the number
  of cosets, then $n|H| = |G|$, so $|H|$ divides $|G|$. 
\end{proof}
\begin{remark}
  If $G$ is finite and $N \trianglelefteq G$, then $|G/N| = |G|/|N|$. 
\end{remark}
\begin{remark}
  For $g_1N \cdot g_2N = (g_1g_2)N$ to hold, $N$ must be normal to $G$ because
  this condition lets us pick any $g_1, g_2 \in N$ and still end up with the
  same coset. 
\end{remark}

\begin{prop}
  Suppose $\phi : G \rightarrow K$ is a homomorphism with $H = \ker \phi$. If
  $h \in H, g \in G$, then $ghg^{-1} \in H$. 
\end{prop}

\begin{nthm}[Algebraic Condition for Normal Subgroups]
  Let $H$ be a subgroup of $G$. Then the following are equivalent:
  \begin{itemize}
  \item $H \trianglelefteq G$
  \item $\forall g \in G, h \in H, ghg^{-1} \in H$
  \end{itemize}
\end{nthm}

\begin{nthm}[First Isomorphism Theorem]
  Let $G, H$ be groups and let $\phi : G \rightarrow H$ be a homomorphism. Then:
  \begin{itemize}
  \item $\ker \phi \trianglelefteq G$
  \item $\img \phi \leq H$
  \item $\img \phi \cong G / \ker(\phi)$
  \end{itemize}
\end{nthm}

\begin{defn}
  Let $X$ be a set and $G$ be a group, and let $x \in X, g \in G$. A \vocab{group action}
  is a binary operation $\cdot : G \times X \rightarrow X$ that sends $x$ go $g \cdot x$.
  It satisfies:
  \begin{itemize}
  \item $\forall x \in X, \forall g_1, g_2 \in G, (g_1g_2) \cdot x = g_1 \cdot (g_2 \cdot x)$
  \item $\forall x \in X, 1_G \cdot x = x$
  \end{itemize}
\end{defn}

\begin{defn}
  Given a group action $G$ on $X$, define an equivalence relation $\sim$ on $X$ such that
  $x \sim y$ if $x = g \cdot y$ for some $g \in G$. The equivalence classes under $\sim$ are
  \vocab{orbits}, denoted $\OO$. In other words, the orbit is everything that can be reached
  from $x$ by an action of something in $G$. 
\end{defn}

\begin{defn}
  The \vocab{stabilizer} of a point $x \in X$, denoted $\operatorname{Stab}_G$, is
  the set of $g \in G$ which fix x. In other words, the set of elements of $G$ which don't
  move when they act on $x$.
  $$\operatorname{Stab}_G(x) = \{g \in G : g \cdot x = x\}$$
\end{defn}

\begin{nthm}[Orbit-Stabilizer Theorem]
  Let $\OO$ be an orbit, and pick any $x \in \OO$. Let $S = \operatorname{Stab}_G(x)$
  be a subgroup of $G$. There is a natural bijection between $\OO$ and left cosets.
  $$|\OO||S| = |G|$$
  In particular, the stabilizers of each $x \in \OO$ have the same size. 
\end{nthm}

\begin{proof}
  Every coset of $gS$ specifies an element of $\OO$, namely $g \cdot x$. Since
  there are $|\OO|$ partitions of $G$ and each one is of size $|S|$,
  the result follows.   
\end{proof}

\begin{nthm}[Burnside's Lemma]
  Let $G$ act on a set $X$. The number of orbits of the action is equal to
  $$\frac{1}{|G|} \sum\limits_{g \in G} |\operatorname{FixPt} g|$$
  where $\operatorname{FixPt} g$ is the set of points $x \in X$ such that
  $g \cdot x = x$. 
\end{nthm}

\begin{defn}
  A common action is \vocab{conjugation}. $G$ acts on itself:
  $$C_G : G \times G \rightarrow G$$
  $$C_G(g, h) = ghg^{-1}$$
\end{defn}

\begin{defn}
  The \vocab{conjugacy classes} of a group $G$ are the orbits of $G$ under the
  conjugacy action. 
\end{defn}

\begin{defn}
  Let $G$ be a group and $H$ be a subgroup of $G$. Let $x \in G$ such that $x \not\in H$.
  $H$ has elements $\{h_1, h_2, ..., h_n\}$. Then the transformation $xh_ix^{-1}$ generates
  the \vocab{conjugate subgroup} $xHx^{-1}$.
  Then if $\forall x, xHx^{-1} = H$, $H$ is a normal subgroup.
\end{defn}

\begin{defn}
  Let $G$ be a group. The \vocab{center} of $G$, denoted $Z(G)$ is the set of
  elements of $x \in G$ such that $xg = gx$ for every $g \in G$. $Z(G)$ is a subgroup
  of $G$. 
  $$Z(G) = \{x \in G : gx = xg \forall g \in G\}$$
\end{defn}

\begin{remark}
  If $G$ is abelian, then the conjugacy classes all have size one. 
\end{remark}

\begin{nthm}[The Sylow Theorems]
  Let $G$ be a group of order $p^nm$ where $\gcd(p, m) = 1$ (in particular,
  $p$ does not divide $m$) and $p$ is prime. A \vocab{Sylow $p$-subgroup} is
  a subgroup of order $p^n$. Let $n_p$ be the number of Sylow $p$-subgroups
  of $G$. Then
  \begin{enumerate}
  \item $n_p \equiv 1 \pmod{p}$. In particular, $n_p \neq 0$ and a Sylow $p$-subgroup exists.
  \item $n_p | m$.
  \item Any two Sylow $p$-subgroups are conjugate subgroups, and hence isomorphic.
  \end{enumerate}
\end{nthm}

\begin{remark}
  All subgroups of an abelian group are normal. 
\end{remark}

\begin{remark}
  Sylow's theorem helps us classify groups:
  \begin{itemize}
  \item A Sylow $p$-subgroup is normal if and only if $n_p = 1$.
  \item Any group $G$ of order $pq$ where $p < q$ are primes must have
    $n_q = 1$, since $n_q \equiv 1 \pmod{q}$ and $n_q | p$. Thus $G$
    has a normal subgroup of order $q$.
  \item If a group $G$ is abelian, for every prime $p$ that divides $|G|$ there
    exists exactly one Sylow $p$-subgroup.
  \item Let $G$ be a group, $P$ be a Sylow $p$-subgroup and $Q$ be a Sylow $q$-subgroup
    (with $p \neq q$ prime). Let $A$ be the intersection of $P$ and $Q$. Then $A$ must
    be a subgroup. Lagrange's Theorem states that $|A|$ divides $|P|$ and $|A|$ divides $|Q|$.
    This is only possible if $|A| = 1$. Since $A$ is a subgroup, $A$ contains only $1_G$.
  \end{itemize}
\end{remark}

\begin{prop}
  If $|G| = pqr$ is the product of distinct primes, then $G$ must have a normal
  Sylow subgroup. 
\end{prop}

\begin{defn}
  Let $G$ be a group and $H$ be a subgroup of $G$. Then the \vocab{normalizer} of $H$
  is the stabilizer of $H$ under conjugation.
  $$N_G(H) = \{g \in G : gHg^{-1} = H\}$$
\end{defn}

\begin{defn}
  A \vocab{simple group} is a group with no normal subgroups other than itself and
  the trivial group.
\end{defn}

\begin{defn}
  A \vocab{composition series} of a group $G$ is a sequence of subgroups $H_0, H_1, ..., H_n$
  such that $H_0 = \{1\}$ and
  $$H_0 \trianglelefteq H_1 \trianglelefteq ... \trianglelefteq H_n$$
  The \vocab{compositional factors} are $H_1/H_0, H_2/H_1, ..., H_n/H_{n-1}$.
\end{defn}

\begin{nthm}[Jordan-Holder Theorem]
  Every finite group $G$ admits a unique composition series up permutation and isomorphism. 
\end{nthm}

\end{document}