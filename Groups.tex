\documentclass{article}
\usepackage{sasha}
\title{Groups}
\author{Sasha Krassovsky}
\begin{document}
\maketitle

\begin{defn}
  A group is a pair $G = (G, \star)$ consisting of a set $G$
  and a binary operation $\star$ on $G$ such that:
  \begin{itemize}
  \item G has an identity $1_G$ or just $1$ such that
    $$\forall g \in G, 1_G \star g = g \star 1_G = g$$
  \item $\star$ is associative, so
    $$\forall a, b, c \in G, (a \star b) \star c = a \star (b \star c)$$
  \item Every element in $G$ has an inverse. That is
    $$\forall g \in G, \exists h \in G, g \star h = h \star g = 1_G$$    
  \end{itemize}
\end{defn}

\begin{remark}
  Notice that $G$ is closed under $\star$ implicitly. That is,
  $\forall g, h \in G, g \star h \in G$. 
\end{remark}

\begin{defn}
  A group is abelian if its operation is commutative ($a \star b = b \star a$).
  Otherwise, it is non-abelian. 
\end{defn}
\noindent Properties of Groups:
\begin{itemize}
\item Let $G$ be a group.
  \begin{enumerate}
  \item The identity $1_G$ is unique.
  \item The inverse of any element $g \in G$, $g^{-1}$ is unique.
  \item For any $g \in G$, $(g^{-1})^{-1} = g$.
  \end{enumerate}
\item Let $G$ be a group and $a, b \in G$. Then $(ab)^{-1} = b^{-1}a^{-1}$.
\item Let $G$ b ea group and pick a $g \in G$. Then the map $G \rightarrow G$ given
  by $x \mapsto gx$ is a bijection. 
\end{itemize}

\begin{defn}
  Let $G = (G, \star)$ and $H = (H, *)$ be groups. A bijection
  $\phi : G \rightarrow H$ is called an isomorphism if
  $$\forall g_1, g_2 \in G, \phi(g_1 \star g_2) = \phi(g_1) * \phi(g_2)$$
  If there exists an isomorphism from $G$ to $H$, then $G$ and $H$ are isomorphic,
  i.e. $G \cong H$. 
\end{defn}

\begin{remark}
  $\cong$ is an equivalence relation (i.e. it is reflexive, symmetric, and transitive). 
\end{remark}

\begin{defn}
  The order of a group $G$ is the number of elements in $G$, denoted $|G|$.
  A group is a finite group if $|G|$ is finite. 
\end{defn}

\begin{defn}
  The order of an element $g \in G$ is the smallest positive integer $n$
  such that $g^n = 1_G$ or $\infty$ if no such $n$ exists. Denoted $\ord{g}$. 
\end{defn}

\end{document}