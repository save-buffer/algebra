\documentclass{article}
\usepackage{sasha}
\title{Rings}
\author{Sasha Krassovsky}
\begin{document}
\maketitle

\begin{defn}
  A \vocab{ring} is a triple $(R, +, \times)$, with operations called addition and
  multiplication. Properties:
  \begin{enumerate}
  \item $(R, +)$ is an abelian group with identity $0_R$.
  \item $\times$ is an associative binary operation on $R$ with an identity $1_R$.
  \item Multiplication distributes over addition.
  \end{enumerate}
  A ring $R$ is \vocab{commutative} if $\times$ is commutative.
\end{defn}

\begin{defn}
  The \vocab{trivial ring} is the ring with only one element, where $0_R = 1_R$. 
\end{defn}

\begin{fact}
  For any ring $R$ with $r \in R$, $r \cdot 0_R$ and $r * (-1_R) = -r$. 
\end{fact}

\begin{defn}
  The \vocab{Gaussian integers} are the complex numbers with integer real and imaginary parts.
  $$Z[i] = \{a + bi : a, b \in \ZZ\}$$
\end{defn}

\begin{defn}
  Let $R, S$ be rings. Then the \vocab{product ring}, denoted $R \times S$ is defined
  as ordered pairs $(r, s)$ with both operations done component-wise. 
\end{defn}

\begin{defn}
  A \vocab{unit} of a ring $R$ is an element $u \in R$ which is invertible. That is,
  for some $x \in R$, $ux = 1_R$. 
\end{defn}

\begin{defn}
  A nontrivial commutative ring is a \vocab{field} if all of its nonzero elements are units. 
\end{defn}
\begin{remark}
  Colloquially, a field is a structure where you can add, subtract, multiply, and divide.
\end{remark}

\begin{defn}
  Let $R$ be a ring and $a, b \in R$ such that $ab = 0$ and $a, b \neq 0$. Then $a$ and
  $b$ are the \vocab{zero divisors} of $R$. \\
  Example: $\ZZ_{15}$ has this property, where $3, 5$ are the zero divisors because
  $3 \cdot 5 \equiv 0 \pmod{15}$. 
\end{defn}

\begin{defn}
  A nontrivial commutative ring with no zero divisors is an \vocab{integral domain}.
\end{defn}

\begin{defn}
  Let $R = (R, +_R, \times_R)$ and $S = (S, +_S, \times_S)$ be rings. A
  \vocab{ring homomorphism} is a map $\phi : R \rightarrow S$ such that
  \begin{enumerate}
  \item $\forall x, y \in R, \phi(x +_R y) = \phi(x) +_S \phi(y)$.
  \item $\forall x, y \in R, \phi(x \times_R y) = \phi(x) \times_S \phi(y)$.
  \item $\phi(1_R) = 1_S$. 
  \end{enumerate}
  If $\phi$ is a bijection, then $\phi$ is an \vocab{isomorphism} and
  $R$ and $S$ are \vocab{isomorphic}.
\end{defn}

\begin{defn}
  The \vocab{kernel} of a ring homomorphism $\phi : R \rightarrow S$, denoted
  $\ker \phi$, is defined as follows:
  $$\ker \phi = \{r \in R : \phi(r) = 0\}$$
\end{defn}

\begin{fact}
  If $\phi(x) = \phi(y) = 0$, then $\phi(x + y) = \phi(x) + \phi(y) = 0$, so
  $\ker \phi$ should be closed under addition. 
\end{fact}

\begin{fact}
  If $\phi(x) = 0$, then $\forall r \in R, \phi(rx) = \phi(r)\phi(x) = 0$, so for
  $x \in \ker \phi$ and any $r \in R$, $rx \in \ker \phi$. 
\end{fact}

\begin{defn}
  A nonempty subset $I \subseteq R$ is an \vocab{ideal} if it is closed under addition,
  and $\forall x \in I, \forall r \in R, rx \in I$.
\end{defn}
\begin{remark}
  This definition is true if we assume the ring is commutative. If the ring is not
  commutative, we must also add the condition that $xr \in I$. 
\end{remark}

\begin{theorem}
  Let $R$ be a ring with $I \subset R$. Then $I$ is the kernel of some homomorphism
  if and only if $I$ is an ideal. 
\end{theorem}

\begin{defn}
  Let $R$ be a ring and $I$ be an ideal. Then the \vocab{quotient ring} is given by
  $$R/I = \{r + I : r \in R\}$$
  $R/I$ is when we declare all elements of $I$ are zero, i.e. we ``mod out by
  elements of $I$''. 
\end{defn}

\begin{theorem}
  The only ideals of $\ZZ$ are those of the form $n\ZZ$ where $n$ is an integer. 
\end{theorem}

\begin{defn}
  Let $R$ be a ring. The \vocab{ideal generated} by a set of elements
  $x_1, ..., x_n \in R$ is denoted $I = (x_1, ..., x_n)$ and given by
  $$I = \{r_1x_1 + ... + r_nx_n : r_i \in R\}$$
  Example: if $I = (x)$, then $I$ consists only of multiples of $x$, i.e. numbers
  of the form $rx$ for $r \in R$. 
\end{defn}

\end{document}