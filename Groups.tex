\documentclass{article}
\usepackage{sasha}
\title{Groups}
\author{Sasha Krassovsky}
\begin{document}
\maketitle

\begin{defn}
  A \vocab{group} is a pair $G = (G, \star)$ consisting of a set $G$
  and a binary operation $\star$ on $G$ such that:
  \begin{itemize}
  \item G has an identity $1_G$ or just $1$ such that
    $$\forall g \in G, 1_G \star g = g \star 1_G = g$$
  \item $\star$ is associative, so
    $$\forall a, b, c \in G, (a \star b) \star c = a \star (b \star c)$$
  \item Every element in $G$ has an inverse. That is
    $$\forall g \in G, \exists h \in G, g \star h = h \star g = 1_G$$    
  \end{itemize}
\end{defn}

\begin{remark}
  Notice that $G$ is closed under $\star$ implicitly. That is,
  $\forall g, h \in G, g \star h \in G$. 
\end{remark}

\begin{defn}
  A group is \vocab{abelian} if its operation is commutative ($a \star b = b \star a$).
  Otherwise, it is \vocab{non-abelian}. 
\end{defn}

\begin{defn}
  A mapping is a \vocab{bijection} if it is injective (one-to-one) and
  surjective (onto). 
\end{defn}

\noindent Properties of Groups:
\begin{itemize}
\item Let $G$ be a group.
  \begin{enumerate}
  \item The identity $1_G$ is unique.
  \item The inverse of any element $g \in G$, $g^{-1}$ is unique.
  \item For any $g \in G$, $(g^{-1})^{-1} = g$.
  \end{enumerate}
\item Let $G$ be a group and $a, b \in G$. Then $(ab)^{-1} = b^{-1}a^{-1}$.
\item Let $G$ b ea group and pick a $g \in G$. Then the map $G \rightarrow G$ given
  by $x \mapsto gx$ is a bijection. 
\end{itemize}

\begin{defn}
  Let $G = (G, \star)$ and $H = (H, *)$ be groups. A bijection
  $\phi : G \rightarrow H$ is called an \vocab{isomorphism} if
  $$\forall g_1, g_2 \in G, \phi(g_1 \star g_2) = \phi(g_1) * \phi(g_2)$$
  If there exists an isomorphism from $G$ to $H$, then $G$ and $H$ are isomorphic,
  i.e. $G \cong H$. 
\end{defn}

\begin{remark}
  $\cong$ is an equivalence relation (i.e. it is reflexive, symmetric, and transitive). 
\end{remark}

\begin{defn}
  The \vocab{order of a group} $G$ is the number of elements in $G$, denoted $|G|$.
  A group is a \vocab{finite group} if $|G|$ is finite.
\end{defn}

\begin{defn}
  The \vocab{order of an element} $g \in G$ is the smallest positive integer $n$
  such that $g^n = 1_G$ or $\infty$ if no such $n$ exists. Denoted $\ord{g}$. 
\end{defn}

\begin{fact}
  If $G^n = 1_G$ then $\ord{g} | n$. 
\end{fact}

\begin{fact}
  Let $G$ be a finite group. Then for all $g \in G, \ord{g}$ is finite.
\end{fact}

\begin{nthm}[Lagrange's Theorem For Orders]
  Let $G$ be any finite group. Then $\forall x \in G, x^{|G|} = 1_G$.
  This is the general case of Fermat's Little Theorem. 
\end{nthm}

\begin{defn}
  The \vocab{General Linear Group} of degree $n$, denoted $\GL_n$ is the set of
  invertible $n \times n$ matrices along with the operation of matrix
  multiplication. To specify what is in each matrix, we give it an argument.
  For example, the GL over $\RR^{n \times n}$ is denoted $\GL_n(\RR)$. 
\end{defn}

\begin{defn}
  Let $G = (G, \star)$ be a group. A group $H = (H, \star)$ is a
  \vocab{subgroup} of $G$ if $H \subseteq G$. $H$ is called a \vocab{proper subgroup}
  of $G$ if $H \neq G$. 
\end{defn}

\begin{remark}
  If $H$ is a subgroup of $G$, the binary operation is the same. Therefore,
  to specify $H$, you need only provide its elements, not its operation. 
\end{remark}

\begin{defn}
  The \vocab{Special Linear Group} of degree $n$ over a field $F$, denoted $\SL_n(F)$
  is the subgroup of $\GL_n(F)$ such that $\forall x \in \SL_n(F), \det(x) = 1$.
\end{defn}

\begin{defn}
  Let $G$ be a group. Let $S$ be a subset of $G$. The subgroup \vocab{generated} by
  $S$, $\langle S \rangle$ is the set of elements which can be written as a finite
  product of elements in $S$ and their inverses. If $\langle S \rangle = G$,
  then $S$ is a set of \vocab{generators} for G. 
\end{defn}

\begin{defn}
  The \vocab{group presentation} of a group is an expression specifying a
  set of generators and \vocab{relations} between the generators. For example,
  $$\ZZ_{100} = \langle x | x^{100} = 1 \rangle$$
\end{defn}

\begin{remark}
  Determining if a group is finite from its presentation is undecideable.
\end{remark}

\begin{defn}
  Let $G = (G, \star)$ and $H = (H, *)$ be groups. A \vocab{group homomorphism}
  is a map $\phi : G \rightarrow H$ such that
  $$\forall g_1, g_2 \in G, \phi(g_1 \star g_2) = \phi(g_1) * \phi(g_2)$$
  Notice that this is is the same as an isomorphism, only lacking the requirement that
  $\phi$ be a bijection. 
\end{defn}

\begin{defn}
  The \vocab{trivial homomorphism} $G \rightarrow H$ sends every element of $G$ to $1_H$.
\end{defn}

\begin{fact}
  Let $\phi : G \rightarrow H$ be a homomorphism. Then $\phi(1_G) = 1_H$ and
  $\phi(g^{-1}) = \phi(g)^{-1}$.
\end{fact}

\begin{defn}
  The \vocab{kernel} of a homomorphism $\phi : G \rightarrow H$ is a subgroup of $G$ such that
  $$\ker \phi = \{g \in G : \phi(g) = 1_H\}$$
\end{defn}

\begin{defn}
  The \vocab{image} of a homomorphism $\phi : G \rightarrow H$ is a subgroup of $G$ such that
  $$\img \phi = \phi(G) = \{\phi(x) : x \in G \}$$
\end{defn}

\begin{prop}
  The map $\phi$ is injective if and only if $\ker \phi = {1_G}$. 
\end{prop}

\begin{defn}
  Let $H$ be any subgroup of $G$. A set of the form $gH$ for any $g \in G$ is called a
  \vocab{left coset} of $H$. 
\end{defn}

\begin{remark}
  It is possible for $g_1N = g_2N$, even if $g_1 \neq g_2$. 
\end{remark}

\begin{remark}
  Given cosets $g_1H$ and $g_2H$, the map $x \mapsto g_2g_1^{-1}$ is a bijection, so
  all cosets have equal cardinality. 
\end{remark}

\begin{defn}
  A subgroup $N$ of $G$ is called \vocab{normal} if it is the kernel of some surjective
  homomorphism. We write this as $N \trianglelefteq G$. 
\end{defn}

\begin{defn}
  Let $N \trianglelefteq G$. Then the \vocab{quotient group}, denoted $G/N$ is
  the group defined such that:
  \begin{itemize}
    \item The elements of $G/N$ will be the left cosets of $N$.
    \item Let $g_1, g_2 \in G$. Then $(g_1N) \cdot (g_2N) = (g_1g_2)N$. 
  \end{itemize}
\end{defn}

\begin{remark}
  We can define an equivalence relation $~_N$ on $G$ by saying $x ~_N y$ for
  $\phi(x) = \phi(y)$. $~_N$ divides $G$ into equivalence classes which are
  in bijection to $G/N$. 
\end{remark}

\begin{nthm}[Lagrange's Theorem]
  Let G be a finite group, and let H be any subgroup. Then $|H|$ divides $|G|$. 
\end{nthm}

\begin{remark}
  If $G$ is finite and $N \trianglelefteq G$, then $|G/N| = |G|/|N|$. 
\end{remark}

\end{document}